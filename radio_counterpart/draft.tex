\documentclass{ltjsarticle}
%==============================================================================%
\usepackage{amsmath,amssymb}
\usepackage{bxpapersize} %用紙サイズを一致させるパッケージ
\usepackage{enumitem}
\usepackage{graphicx}
\usepackage[unicode=true]{hyperref}
\usepackage{mathtools}
\mathtoolsset{showonlyrefs}
\usepackage{siunitx, letltxmacro}
\AtBeginDocument{\RenewCommandCopy\qty\SI}
\LetLtxMacro{\svqty}{\qty}
\usepackage{physics}
\LetLtxMacro{\qty}{\svqty}
\usepackage{booktabs}
\usepackage{empheq}
\usepackage{titleps}
\newpagestyle{main}{
  \sethead[][][]
    {}{}{}
  \setfoot[][\thepage][]
    {}{\thepage}{}
}
\pagestyle{main}
\AtBeginDocument{\RenewCommandCopy\qty\SI}
%==============================================================================%
\newcommand{\Edd}{\mathrm{Edd}}
\newcommand{\Mdot}{\dot{M}}

\newcommand{\fMx}{M_{*,\num{1.4}}}
\newcommand{\fepsB}{\epsilon_{\mathrm{B},\num{0.01}}}
\newcommand{\feps}{\epsilon_{\num{0.1}}}
\newcommand{\ftrec}{t_{\mathrm{rec},\num{200}}}
\newcommand{\ftrise}{t_{\mathrm{rise},\num{3}}}
\newcommand{\ftdur}{t_{\mathrm{X},\num{10}}}
\newcommand{\ftdecay}{\tau_{\num{10}}}
\newcommand{\fvsh}{v_{\mathrm{sh},\num{0.3}}}
\newcommand{\fzeta}{\zeta_{\num{0.4}}}
\newcommand{\fdis}{d_{\num{4.5}}}

\newcommand{\rsh}{r_\mathrm{sh}}
\newcommand{\lsh}{l_\mathrm{sh}}
\newcommand{\vsh}{v_\mathrm{sh}}

\begin{document}
  % \title{Assignment on 2024-10-29}
  % \author{KUMADA Ryota (C4SM3002)}
  % \date{}
  % \maketitle
%------------------------------------------------------------------------------%
% \section{電波放射するまでの流れ}
% \begin{enumerate}
%   \item X--ray burstでshockが形成される
%   \item shockが外側の物質を加熱する
% \end{enumerate}
% \section{Preface}
% Type 1 X--ray bursterとして知られている4U 1728--34から電波放射が観測された.
% ここでは, 設定した電波放射モデルが電波放射を再現できるかどうかについて議論する.
% 4U 1728--34の観測値は
% \url{https://binry-revolution.github.io/LMXBwebcat/LMXBs/NAME%20Slow%20Burster.html}
% に記載されている.
\section{Accretion rate}
Eddington luminosity~$L_\Edd$は以下で定義される:
\begin{equation}
  L_\Edd
  = \frac{4\pi G M_* m_p c}{\sigma_\mathrm{T}}
  = \SI{1.26e38}{erg/s}~\left( \frac{M_*}{M_\odot} \right)
\end{equation}
\( G,\,M_*,\,m_p,\,c,\,\sigma_\mathrm{T} \)はそれぞれ
重力定数, 中性子星の質量, 陽子の質量, 真空中の光速, トムソン散乱断面積.

Eddington accretion rate~$\Mdot_\Edd$を以下で定義する:
\begin{empheq}{align}
  \Mdot_\Edd \coloneqq L_\Edd / c^2
  &= \SI{1.34e17}{g/s}~\left(\frac{M_*}{M_\odot}\right) \\
  &= \num{2.22e-9}~M_\odot / \si{yr} ~ \left(\frac{M_*}{M_\odot}\right)
\end{empheq}

4U 1728--34のfiducial valueとして, mass~\( M_* = 1.4M_\odot \)(Shaposhnikov et al.~2003),
accretion rate~\(\Mdot = 0.1\Mdot_\Edd \)(Galloway et al.~2008)を, それぞれ採用する.
すなわち,
\begin{equation}
  \Mdot = \epsilon\Mdot_\Edd(1.4M_\odot)
  = \SI{1.96e16}{g/s}~\feps~\fMx
\end{equation}

\section{Shockの形成}
X--ray burstのduration time~\( t_X \approx \SI{10}{s} \)の間に飛んでいったモノが
shock ejectaを形成すると仮定する.
Ejectaの運動する速さを\(\vsh = 0.3c\)とすると, shockの厚み$\lsh$は
\begin{equation}
  \lsh \approx v_\mathrm{sh} t_X
  =
  \SI{9e10}{cm}~\fvsh~\ftdur
\end{equation}

shock ejectaの質量$M_\mathrm{ej}$は, recurrence time \( t_\mathrm{rec} \approx \SI{200}{min} \)の
間に降着する質量の$\eta$倍に等しいとすると
\begin{equation}
  M_\mathrm{ej}
  \approx
  \eta\Mdot t_\mathrm{rec}
  =
  \SI{2.3e20}{g}~\eta~\feps~\fMx~\ftrec
\end{equation}

% 観測によれば, X--ray burstが起きてから電波が立ち上がるまでの時間は
% \( t_\mathrm{rise} \approx \SI{3}{min} \)である.
電波が立ち上がるのは, X--ray burstが起きてから
\( t_\mathrm{rise} \approx \SI{3}{min} \)後である.
shockが形成される位置$\rsh$は
\begin{equation}
  \rsh \approx v_\mathrm{sh}t_\mathrm{rise} 
  =
  \SI{1.6e12}{cm}~\fvsh~\ftrise
  =
  \SI{0.11}{AU}~\fvsh~\ftrise
\end{equation}

\section{磁場の強さ}
collisionless--shockはそのエネルギーの一部を磁場へ渡す. 変換効率を$\epsilon_B$とすると,
エネルギー密度について
\begin{equation}
  \epsilon_\mathrm{B}
  \frac{M_\mathrm{ej}v^2_\mathrm{sh}/2}{4\pi r^2_\mathrm{sh}\lsh}
  =
  \frac{B^2}{8\pi}
\end{equation}
% ここで\( \tau \approx \SI{10}{min} \)はX--ray burstが起きてから電波が減衰するまでの時間である.
$B$について解くと
\begin{empheq}{align}
  B^2
  &=
  \frac{\epsilon_\mathrm{B}M_\mathrm{ej}v_\mathrm{sh}^2}{\rsh^2\lsh}
  =
  \SI{8.1e2}{erg.cm^{-3}}~
  \eta~\feps~\fepsB~\fMx~\fvsh^{-1}~\ftrec~\ftrise^{-1}~\ftdur^{-1}
\end{empheq}
\begin{equation}
  B
  \approx
  \SI{28}{G}~
  \sqrt{\eta}~\feps^{1/2}~\fepsB^{1/2}~\fMx^{1/2}~\fvsh^{-1/2}~
  \ftrec^{1/2}~\ftrise^{-1/2}~\ftdur^{-1/2}
  % \sqrt{\eta}~\fepsB^{1/2}~\feps^{1/2}~\fMx^{1/2}~
  % \fvsh^{-1/2}~\ftrec^{1/2}~\ftrise^{-1}~\ftdecay^{-1/2}
\end{equation}
$\epsilon_\mathrm{B} = 0.01$はKashiyama et al.(2018)から引用.

\section{周辺ガスの密度}
電波は\( \tau \approx \SI{10}{min} \)の時間で減衰する.
この間にshockは\( 4\pi \rsh^2\vsh\tau \)の領域を掃く.
掃過領域にあるガス(一様密度$\rho_\mathrm{amb}$を仮定)の全質量が$M_\mathrm{ej}$に等しいとすると
\begin{equation}
  4\pi r^2_\mathrm{sh}\rho_\mathrm{amb}v_\mathrm{sh}\tau = M_\mathrm{ej}
\end{equation}
を満たす. $\rho_\mathrm{amb}$について解くと
\begin{empheq}{align}
  \rho_\mathrm{amb}
  &=
  \frac{M_\mathrm{ej}}{4\pi r^2_\mathrm{sh}v_\mathrm{sh}\tau}\\
  &=
  \SI{1.3e-18}{g.cm^{-3}}
  \feps~\fMx~\fvsh^{-3}~\ftrec~\ftrise^{-2}~\ftdecay^{-1}
  % \epsilon_1 ~
  % (M_{*,\num{1.4}})~
  % (t_{\mathrm{rec},\num{200}})~
  % (v_{\mathrm{sh},\num{0.3}})^{-3}~
  % (t_{\mathrm{rise},\num{3}})^{-2}~
  % (\tau_{\num{10}})^{-1}~
\end{empheq}

ガスがすべて水素で構成されているとすると, 電子数密度は
\begin{equation}
  n_\mathrm{amb,e}
  = \rho_\mathrm{amb} / (1\cdot m_u)
  = \SI{7e5}{cm^{-3}}~
  \feps~\fMx~\fvsh^{-3}~\ftrec~\ftrise^{-2}~\ftdecay^{-1}
\end{equation}

\section{電子のLorentz factor}
\subsection{Maximum Lorentz factor}
電子の最大Lorentz factor$\gamma_\mathrm{max}$は
\begin{equation}
  \left(\frac{9\pi e\beta^2}{10\sigma_\mathrm{T}B}\right)^{1/2} \approx
  \left(\frac{9\pi e}{10\sigma_\mathrm{T}B}\right)^{1/2} \:(\beta\to 1)
\end{equation}
で与えられる(Kashiyama et al.(2018)). 値を代入すると
\begin{equation}
  \gamma_\mathrm{max} = \num{8.5e6}~B_{\num{28}}^{-1/2}
\end{equation}

\subsection{Minimum Lorentz factor}
プラズマガスはshockからエネルギーを受け取り熱化される. 電子の熱エネルギー$kT_e$が, $v_\mathrm{sh}$で走る電子の運動エネルギーに等しいと仮定すると
\begin{equation}
  kT_e
  = \frac{m_e}{2}v_\mathrm{sh}^2
  = 0.045 m_e c^2 ~\fvsh^2
\end{equation}

プラズマが電子と陽子のみから構成されていると仮定する. 陽子の熱化についても電子と同様に考えると,
各々のLorentz factorはいずれも
\begin{equation}
  \gamma_\mathrm{th} = \frac{kT_i}{m_i c^2} \approx 0.045\quad (i = \mathrm{e,\,p})
\end{equation}
となる.

熱化された陽子と電子とが互いにエネルギーをやりとりできるのであれば,
電子が陽子の熱エネルギーの一部を受け取れるはずである.
電子が受け取る, 陽子の熱エネルギーの割合を$\zeta$とする.
電子のLorentz factorは, もともとの熱エネルギーに陽子からもらった分が加わると考えて
\begin{empheq}{align}
  \gamma_\mathrm{min} 
  &=
  \frac{1}{m_e c^2} ( kT_e + \zeta kT_p )\\
  &\approx
  \zeta \frac{kT_p}{m_e c^2} \:(\text{ignoring}\:kT_e)\\
  &=
  \zeta \frac{m_p}{2m_e}\frac{v_\mathrm{sh}^2}{c^2} \\
  &=
  33~\fzeta~\fvsh^{2}
\end{empheq}
となる. 
$\zeta= 0.4$はKashiyama et al.(2018)の値を引用.

\section{シンクロトロン放射}
$\gamma_e$のLorenz factorをもつ電子からのシンクロトロン放射を考える.
放射の典型的な振動数$\nu_e$は
\begin{equation}
  \nu_e
  = \gamma_e^3\nu_B\sin\alpha
  = \frac{\gamma_e^2eB}{2\pi m_e c} \sin\alpha
\end{equation}
である(Rybicki \& Lightman (1979)).
ここで$\nu_B$はサイクロトロン振動数, $\alpha$はピッチ角である.
磁場がランダムな方向を向いていると仮定し, $\sin\alpha$の平均をとると
\begin{equation}
  \int_0^{\pi/2} \sin\alpha \dd{\alpha} = 1
\end{equation}
ゆえ
\begin{equation}
  \nu_e = \frac{\gamma_e^2eB}{2\pi m_e c}
\end{equation}
\( \gamma_e = \gamma_\mathrm{min} \)として値を代入すると
\begin{equation}
  \nu_e
  \approx
  \SI{87}{GHz}~
  \sqrt{\eta}~
  \feps^{1/2}~\fepsB^{1/2}~\fzeta^2~\fMx^{1/2}~
  \fvsh^{3/2}~\ftrec^{1/2}
\end{equation}
% ひとまずKashiyama et al.(2018)にならって\( \zeta = 0.4,\,\epsilon_\mathrm{B} = 0.01 \)
% を入れてみると
% \begin{equation}
%   \nu_e \approx \SI{7.92}{GHz}
% \end{equation}

シンクロトロン放射した電子はその分エネルギーを失う.
あるLorentz factor$\gamma$をもつ電子がエネルギー密度$U_B$の磁場の中で単位時間に放射するエネルギー
(=電子のエネルギー損失率)は
\begin{equation}
  P(\gamma,B) = \frac{4}{3}\sigma_\mathrm{T}c\beta^2\gamma^2U_B
\end{equation}
と表せる(Rybicki\&Lightman, 1979).
いまの場合,
\begin{equation}
  \gamma = \gamma_\mathrm{min} = 33~\fzeta~\fvsh^{2}\:,
\end{equation}
\begin{equation}
  \beta^2 = 1 - \frac{1}{\gamma_\mathrm{min}^2} ~(=0.999)~ \approx 1
\end{equation}
\begin{equation}
  U_B
  =
  \frac{B^2}{8\pi}
  % =
  % \epsilon_\mathrm{B}
  % \frac{M_\mathrm{ej}v^2_\mathrm{sh}/2}{4\pi r^2_\mathrm{sh}v_\mathrm{sh}\tau}
  = \SI{32}{erg.em^{-3}}~
  \eta~\feps~\fepsB~\fMx~\fvsh^{-1}~\ftrec~\ftrise^{-1}~\ftdur^{-1}
  % \eta~\fepsB~\feps~\fMx~\fvsh^{-1}~\ftrec~\ftrise^{-2}~\ftdecay^{-1}
\end{equation}
なので
\begin{empheq}{align}
  P &= \SI{9.3e-10}{erg/s}~
  \eta~\feps~\fepsB~\fzeta^2\fMx~\fvsh^3~\ftrec~\ftrise^{-1}~\ftdur^{-1}\\
  &= \SI{0.58}{keV/s}~
  \eta~\feps~\fepsB~\fzeta^2\fMx~\fvsh^3~\ftrec~\ftrise^{-1}~\ftdur^{-1}\\
\end{empheq}
となる.
よって, シンクロトロン放射の寄与だけを考えた冷却時間は
\begin{equation}
  t_\mathrm{cool}
  \approx
  \frac{\gamma_\mathrm{min}m_ec^2}{P}
  \approx
  \SI{484}{min}
  ~\eta^{-1}~\fepsB^{-1}~\feps^{-1}~\fzeta^{-1}~\fMx^{-1}
  ~\fvsh^{-2}~\ftrec^{-1}~\ftrise~\ftdur
\end{equation}

上で求めた$P$は1個の電子による放射パワーである.
shockで掃かれたガスに含まれる電子がすべて
$\gamma_\mathrm{min}$のシンクロトロン放射しているならば, 
% ビーミングを考慮した上での
電子の総数$N_e$は
\begin{equation}
  N_e \approx \frac{M_\mathrm{ej}}{m_u}
  =
  \num{1.4e44}~
  \eta~\feps~\fMx~\ftrec
\end{equation}
% \begin{equation}
%   N_e \approx \frac{M_\mathrm{ej}}{m_u} \times \frac{2\gamma_\mathrm{min}^{-1}}{2\pi}
%   =
%   \num{1.4e42}~
%   \eta~\feps~\fzeta~\fMx~\fvsh^2~\ftrec
% \end{equation}
となる.
\begin{equation}
  P_\mathrm{tot}
  =
  N_e P
  =
  \SI{1.3e35}{erg/s}~
  \eta^2~\feps^2~\fepsB~\fzeta^2~\fMx^2~\fvsh^3~\ftrec^2~\ftrise^{-1}~\ftdur^{-1}
\end{equation}

% 4U 1728--34までの距離\( d \approx \SI{4.5}{kpc} = \SI{1.4e22}{cm} \)
% (Shaposhnikov et al.(2003))を用いると, fluxは
% \begin{empheq}{align}
%   F_\nu
%   &\approx P_\mathrm{tot} / (4\pi d^2 \nu_e)\\
%   &=
%   \SI{6e2}{mJy}~
%   \eta^{1/2}~\feps^{3/2}~\fepsB^{1/2}~\fzeta^{-1}~\fMx^{3/2}~
%   \fvsh^{7/2}~\ftrec^{3/2}~\ftrise^{-1}~\ftdur^{-1}~\fdis^{-2}
% \end{empheq}
% となる. (Rassell et al.(2024)の100倍)


\section{Power--lawに従う電子分布からのシンクロトロン放射}
\subsection{比例係数の決定}
電子分布はpower lawに従うとする.
\begin{equation}
  \label{eq:powerlaw_gamma}
  \dv{N}{\gamma} = N_0\gamma^{-p}\quad(\gamma_\mathrm{min} < \gamma < \gamma_\mathrm{max})
\end{equation}
ここで$N_0$は数密度の次元をもつ定数である.

shockはそのエネルギーの一部を周囲の電子へ受け渡す. 効率を$\epsilon_\mathrm{e}$とおくと
\begin{empheq}{align}
  \epsilon_\mathrm{e} 
  \frac{M_\mathrm{ej}v^2_\mathrm{sh}/2}{4\pi\rsh^2\lsh}
  &=
  \int_{\gamma_\mathrm{min}}^{\gamma_\mathrm{max}}
  \gamma m_\mathrm{e}c^2 \dv{N}{\gamma}\dd{\gamma} \\
  &=
  m_\mathrm{e}c^2 \frac{N_0}{2-p}
  \left[ \gamma^{2-p} \right]_{\gamma_\mathrm{min}}^{\gamma_\mathrm{max}} \\
  &\approx
  m_\mathrm{e}c^2 \frac{N_0}{p-2} \gamma_{\mathrm{min}}^{2-p}
  \quad (p>2,\,\gamma_{\mathrm{max}} \gg \gamma_{\mathrm{min}})
\end{empheq}
が成り立つ. $N_0$について解くと
\begin{empheq}{align}
  N_0
  &=
  (p-2)\gamma_{\mathrm{min}}^{p-2}~
  \frac{\epsilon_e M_\mathrm{ej}\vsh^2/2}{4\pi\rsh^2\lsh m_e c^2} \\
  &=
  \SI{1.1e9}{cm^{-3}}~ \epsilon_e (p-2) \gamma_\mathrm{min}^{p-2}
  % \num{1.1e46}~(p-2)~\gamma_{\mathrm{min}}^{p-2}\epsilon_e
\end{empheq}

\subsection{Power--lawに従う電子分布からのシンクロトロン放射}
分布が\eqref{eq:powerlaw_gamma}で与えられる電子のシンクロトロン放射スペクトルは
以下で与えられる(Rybicki \& Lightman, 1979):
\begin{empheq}{align}
  P_{\nu} 
  &= 
  \frac{\sqrt{3}e^3N_0B}{m_ec^2(p+1)}
  \Gamma\left( \frac{p}{4} + \frac{19}{12} \right)
  \Gamma\left( \frac{p}{4} - \frac{1}{12} \right)
  \left( \frac{2\pi\nu m_ec}{3eB} \right)^{-(p-1)/2} \\
  &=
  \frac{eB\hbar c}{m_e c^2}
  \left( \frac{2\pi m_e c^2 \nu}{3eBc} \right)^{-(p-1)/2}
  \frac{\sqrt{3}\alpha N_0}{p+1}
  \Gamma\left( \frac{3p+19}{12} \right)
  \Gamma\left( \frac{3p-1}{12} \right) \\
  &=
  \frac{\sqrt{3}N_0 r_e f}{p+1}
  \left( \frac{2\pi E_\mathrm{res}\lambda^{-1}}{3f} \right)^{-(p-1)/2}
  \Gamma\left( \frac{3p+19}{12} \right)
  \Gamma\left( \frac{3p-1}{12} \right) \\
\end{empheq}
ここで, 
\( r_e = e^2/(m_e c^2) \)は古典電子半径である.
また, \( f = eB,\, E_{\mathrm{res}} = m_e c^2,\, \lambda = c/\nu\)
とおいた.

\subsection{SSAの吸収係数}
エネルギー$E$あたりの電子の密度分布\( N(E) \)は以下に従うと仮定する.
\begin{equation}
  \dv{N(E)}{E} = \tilde{N}_0 E^{-p}
\end{equation}
\( \gamma = E/(m_e c^2) \)なので,
$\tilde{N}_0$と$N_0$とは以下の関係がある:
\begin{equation}\label{eq:powerlaw_convert_NE_Ngamma}
  \tilde{N}_0 = (m_e c^2)^{p-1} N_0
\end{equation}

synchrotron self--absorptionの
吸収係数は以下で与えられる(Rybicki \& Lightman, 1979).
\begin{empheq}{align}
  \alpha_\nu
  &=
  \frac{\sqrt{3}e^3}{8\pi m_e} \left( \frac{3e}{2\pi m_e^3 c^5} \right)^{p/2}
  \tilde{N}_0B^{(p+2)/2}~
  \Gamma\left( \frac{3p+2}{12} \right)
  \Gamma\left( \frac{3p+22}{12} \right)
  \nu^{-(p+4)/2} \\
  &=
  \frac{\sqrt{3}\tilde{N}_0 r_e}{8\pi}
  eB\left( \frac{c}{\nu} \right)^2
  \left( \frac{3}{2\pi}\frac{eB}{(m_e c^2)^3}
  \frac{c}{\nu} \right)^{p/2}
  \Gamma\left( \frac{3p+2}{12} \right)
  \Gamma\left( \frac{3p+22}{12} \right) \\
  &=
  \frac{\sqrt{3}{E_\mathrm{res}}^{p-1}N_0 r_e f}{8\pi \lambda^{-2}}
  \left( \frac{3f}{2\pi E_{\mathrm{res}}^3 \lambda^{-1}} \right)^{p/2}
  \Gamma\left( \frac{3p+2}{12} \right)
  \Gamma\left( \frac{3p+22}{12} \right)
\end{empheq}
最後の等式で\eqref{eq:powerlaw_convert_NE_Ngamma}を用いた.

源泉関数は
\begin{empheq}{align}
  S_\nu
  \coloneqq \frac{P_\nu}{4\pi\alpha_\nu}
  &=
  \frac{2\lambda^{-2} E_\mathrm{res}}{p+1}
  \left( \frac{2\pi E_\mathrm{res}\lambda^{-1}}{3f} \right)^{1/2}
  G(p) \\
  &=
  \left( \frac{\nu}{c} \right)^{5/2}
  \frac{2m_e c^2}{p+1}
  \left( \frac{2\pi m_e c^2}{3eB} \right)^{1/2}
  G(p)
\end{empheq}
となる. ただし
\begin{equation}
  G(p) = 
  \frac{
    \Gamma\left( \frac{3p+19}{12} \right)
    \Gamma\left( \frac{3p-1}{12} \right)
  }{
  \Gamma\left( \frac{3p+2}{12} \right)
  \Gamma\left( \frac{3p+22}{12} \right)
  }
\end{equation}
である.

天体までの距離\( d = \SI{4.5}{kpc} \)とすると, 観測されるflux~$F_\nu$は
\begin{equation}
  F_\nu
  \approx S_\nu\left( \frac{\rsh^2}{d^2} \right)
  \approx \num{1.4e-20}S_\nu
\end{equation}
となる.

\subsection{Optical depthの評価}
Optical depth~$\tau_\nu$は
\begin{equation}
  \tau_\nu \coloneqq \alpha_\nu \lsh
  =
  \frac{\sqrt{3} (m_e c^2)^{p-1} N_0 r_e eB \lsh}{8\pi (\nu/c)^2}
    \left( \frac{3eB}{2\pi(m_e c^2)^3(\nu/c)} \right)^{p/2}
    \Gamma\left( \frac{3p+2}{12} \right)
    \Gamma\left( \frac{3p+22}{12} \right)
\end{equation}
とかける. \( \tau = 1 \)のときの周波数$\nu_0$を求めると
\begin{equation}
  \nu_0/c
  =
  \left( 
    \frac{\sqrt{3} (m_e c^2)^{p-1} N_0 r_e eB \lsh}{8\pi}
    \Gamma\left( \frac{3p+2}{12} \right)
    \Gamma\left( \frac{3p+22}{12} \right)
  \right)^{2/(p+4)}
  \left( \frac{3eB}{2\pi(m_e c^2)^3} \right)^{p/(p+4)}
\end{equation}
\end{document}
