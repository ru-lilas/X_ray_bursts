\documentclass{ltjsarticle}
%==============================================================================%
\usepackage{amsmath,amssymb}
\usepackage{bxpapersize} %用紙サイズを一致させるパッケージ
\usepackage{enumitem}
\usepackage{graphicx}
\usepackage[unicode=true]{hyperref}
\usepackage{mathtools}
\mathtoolsset{showonlyrefs}
\usepackage{siunitx, letltxmacro}
\AtBeginDocument{\RenewCommandCopy\qty\SI}
\LetLtxMacro{\svqty}{\qty}
\usepackage{physics}
\LetLtxMacro{\qty}{\svqty}
\usepackage{booktabs}
\usepackage{empheq}
\usepackage{titleps}
\newpagestyle{main}{
  \sethead[][][]
    {}{}{}
  \setfoot[][\thepage][]
    {}{\thepage}{}
}
\pagestyle{main}
\AtBeginDocument{\RenewCommandCopy\qty\SI}
%==============================================================================%
\newcommand{\Edd}{\mathrm{Edd}}
\newcommand{\Mdot}{\dot{M}}

\newcommand{\fMx}{M_{*,\num{1.4}}}
\newcommand{\fepsB}{\epsilon_{\mathrm{B},\num{0.01}}}
\newcommand{\feps}{\epsilon_{\num{0.1}}}
\newcommand{\ftrec}{t_{\mathrm{rec},\num{200}}}
\newcommand{\ftrise}{t_{\mathrm{rise},\num{3}}}
\newcommand{\ftdecay}{\tau_{\num{10}}}
\newcommand{\fvsh}{v_{\mathrm{sh},\num{0.3}}}
\newcommand{\fzeta}{\zeta_{\num{0.4}}}
\begin{document}
  % \title{Assignment on 2024-10-29}
  % \author{KUMADA Ryota (C4SM3002)}
  % \date{}
  % \maketitle
%------------------------------------------------------------------------------%
% \section{電波放射するまでの流れ}
% \begin{enumerate}
%   \item X--ray burstでshockが形成される
%   \item shockが外側の物質を加熱する
% \end{enumerate}
% \section{Preface}
% Type 1 X--ray bursterとして知られている4U 1728--34から電波放射が観測された.
% ここでは, 設定した電波放射モデルが電波放射を再現できるかどうかについて議論する.
% 4U 1728--34の観測値は
% \url{https://binry-revolution.github.io/LMXBwebcat/LMXBs/NAME%20Slow%20Burster.html}
% に記載されている.
\section{Eddington accretion rate}
Eddington luminosity~$L_\Edd$は以下で定義される:
\begin{equation}
  L_\Edd
  = \frac{4\pi G M_* m_p c}{\sigma_\mathrm{T}}
  = \SI{1.26e38}{erg/s}~\left( \frac{M_*}{M_\odot} \right)
\end{equation}
\( G,\,M_*,\,m_p,\,c,\,\sigma_\mathrm{T} \)はそれぞれ
重力定数, 中性子星の質量, 陽子の質量, 真空中の光速, トムソン散乱断面積.

Eddington accretion rate~$\Mdot_\Edd$を以下で定義する:
\begin{empheq}{align}
  \Mdot_\Edd \coloneqq L_\Edd / c^2
  &= \SI{1.34e17}{g/s}~\left(\frac{M_*}{M_\odot}\right) \\
  &= \num{2.22e-9}~M_\odot / \si{yr} ~ \left(\frac{M_*}{M_\odot}\right)
\end{empheq}
\section{Accretion rateとejecta mass}
4U 1728--34のfiducial valueとして, mass~\( M_* = 1.4M_\odot \)(Shaposhnikov et al.~2003),
accretion rate~\(\Mdot = 0.1\Mdot_\Edd \)(Galloway et al.~2008)を, それぞれ採用する.
すなわち,
\begin{equation}
  \Mdot = \epsilon\Mdot_\Edd(1.4M_\odot)
  = \SI{1.96e16}{g/s}~\feps\fMx
\end{equation}

shock ejectaの質量$M_\mathrm{ej}$は, recurrence time \( t_\mathrm{rec} \approx \SI{200}{min} \)の
間に降着する質量の$\eta$倍に等しいとする.
\begin{equation}
  M_\mathrm{ej}
  =
  \eta\Mdot t_\mathrm{rec}
  =
  \SI{2.35e20}{g}~\eta\feps\fMx\ftrec
\end{equation}

\section{電波立ち上がり時のshock radius}
Shock ejectaの速さを\(v_\mathrm{sh} = 0.3c\)と仮定する.
観測によれば, X--ray burstが起きてから電波が立ち上がるまでの時間は
\( t_\mathrm{rise} \approx \SI{3}{min} \)である.
shockが形成される位置$r_\mathrm{sh}$を見積もると
\begin{equation}
  r_\mathrm{sh} = v_\mathrm{sh}t_\mathrm{rise} 
  =
  \SI{1.6e12}{cm}~\fvsh\ftrise
  =
  \SI{0.11}{AU}~\fvsh\ftrise
\end{equation}

\section{磁場の強さ}
collisionless--shockはそのエネルギーの一部を磁場へ渡す. 変換効率を$\epsilon_B$とすると
\begin{equation}
  \epsilon_\mathrm{B}\frac{M_\mathrm{ej}v^2_\mathrm{sh}/2}{4\pi r^2_\mathrm{sh}v_\mathrm{sh}\tau}
  =
  \frac{B^2}{8\pi}
\end{equation}
ここで\( \tau \approx \SI{10}{min} \)はX--ray burstが起きてから電波が減衰するまでの時間である.
$B$について解くと
\begin{empheq}{align}
  B^2
  &=
  \epsilon_B\frac{M_\mathrm{ej}v_\mathrm{sh}}{r^2_\mathrm{sh}\tau}
  =
  \eta\epsilon_B
  \frac{\Mdot t_\mathrm{rec}}{v_\mathrm{sh} t^2_\mathrm{rise}\tau}\\
  &\approx
  \SI{13.4}{erg.cm^{-3}}~
  \eta~\fepsB~\feps~\fMx~\fvsh^{-1}~\ftrec~\ftrise^{-2}~\ftdecay^{-1}
  % \epsilon^{\mathrm{B}}_{\num{1}}~
  % \epsilon_1 ~
  % M^*_{\num{4}}~
  % t^\mathrm{rec}_{\num{200}}~
  % (v^\mathrm{sh}_{\num{3}})^{-1}~
  % (t^\mathrm{rise}_{\num{3}})^{-2}~
  % (\tau_{\num{10}})^{-1}~
  % \frac{\tau}{\SI{10}{min}}
\end{empheq}
\begin{equation}
  B
  \approx
  \SI{3.66}{G}~
  \sqrt{\eta}~\fepsB^{1/2}~\feps^{1/2}~\fMx^{1/2}~
  \fvsh^{-1/2}~\ftrec^{1/2}~\ftrise^{-1}~\ftdecay^{-1/2}
\end{equation}
$\epsilon_\mathrm{B} = 0.01$はKashiyama et al.(2018)をそのまま引用.

\section{周辺ガスの密度}
\( \tau \approx \SI{10}{min} \)の間にshockが掃いた領域にあるガスの総質量は$M_\mathrm{ej}$に等しいと仮定する. すなわち
\begin{equation}
  4\pi r^2_\mathrm{sh}\rho_\mathrm{amb}v_\mathrm{sh}\tau = M_\mathrm{ej}
\end{equation}
$\rho_\mathrm{amb}$は周辺ガスの質量密度である. $\rho_\mathrm{amb}$について解くと
\begin{empheq}{align}
  \rho_\mathrm{amb}
  &=
  \frac{M_\mathrm{ej}}{4\pi r^2_\mathrm{sh}v_\mathrm{sh}\tau}\\
  &\approx
  \SI{1.32e-18}{g.cm^{-3}}
  \feps~\fMx~\fvsh^{-3}~\ftrec~\ftrise^{-2}~\ftdecay^{-1}
  % \epsilon_1 ~
  % (M_{*,\num{1.4}})~
  % (t_{\mathrm{rec},\num{200}})~
  % (v_{\mathrm{sh},\num{0.3}})^{-3}~
  % (t_{\mathrm{rise},\num{3}})^{-2}~
  % (\tau_{\num{10}})^{-1}~
\end{empheq}

\section{Power--lawの比例係数}
工事中.
% 電子分布はpower lawに従うとする.
% \begin{equation}
%   \dv{N}{\gamma} = C\gamma^{-p}\quad(\text{for}\;\gamma_\mathrm{min} < \gamma < \gamma_\mathrm{max})
% \end{equation}
% shockはそのエネルギーの一部を周囲の電子へ受け渡す. 効率を$\epsilon_\mathrm{e}$とおくと
% \begin{equation}
%   \epsilon_\mathrm{e} \frac{M_\mathrm{ej}}{2}v^2_\mathrm{sh}
%   =
%   \int_{\gamma_\mathrm{min}}^{\gamma_\mathrm{max}} \gamma m_\mathrm{e}c^2 \dv{N}{\gamma}\dd{\gamma}
%   =
%   m_\mathrm{e}c^2 \frac{C}{2-p}
%   \left[ \gamma^{2-p} \right]_{\gamma_\mathrm{min}}^{\gamma_\mathrm{max}}
% \end{equation}
% となる. $C$について解くと
% \begin{empheq}{align}
%   C
%   &=
%   \epsilon_\mathrm{e} \frac{M_\mathrm{ej}v^2_\mathrm{sh}}{2m_\mathrm{e}c^2}
%   (2-p) \left( \gamma_\mathrm{max}^{2-p} - \gamma_\mathrm{min}^{2-p} \right)^{-1}\\
%   &=
%   \num{6.6e46}\eta\epsilon_\mathrm{e}~
%   (2-p) \left( \gamma_\mathrm{max}^{2-p} - \gamma_\mathrm{min}^{2-p} \right)^{-1}
%   \left( \frac{\dot{M}}{\dot{M}_\mathrm{Edd}} \right)
%   \left( \frac{t_\mathrm{rec}}{\SI{200}{min}} \right)
%   \left( \frac{v_\mathrm{sh}}{0.3c} \right)^2
% \end{empheq}

% shockで加速される電子のエネルギー範囲($\gamma_\mathrm{min},\,\gamma_\mathrm{max}$)が
% 分かれば, $C$の値を見積もることができる.

\section{放射電子の最小エネルギー}
プラズマガスはshockからエネルギーを受け取り熱化される. 電子の熱エネルギー$kT_e$が, $v_\mathrm{sh}$で走る電子の運動エネルギーに等しいと仮定すると
\begin{equation}
  kT_e
  = \frac{m_e}{2}v_\mathrm{sh}^2
  = 0.045 m_e c^2 ~\fvsh^2
\end{equation}

プラズマが電子と陽子のみから構成されていると仮定する. 陽子の熱化についても電子と同様に考えると,
各々のLorentz factorはいずれも
\begin{equation}
  \gamma_\mathrm{th} = \frac{kT_i}{m_i c^2} \approx 0.045\quad (i = \mathrm{e,\,p})
\end{equation}
となる.

熱化された陽子と電子とが互いにエネルギーをやりとりできるのであれば,
電子が陽子の熱エネルギーの一部を受け取れるはずである.
電子が受け取る, 陽子の熱エネルギーの割合を$\zeta$とする.
電子のLorentz factorは, もともとの熱エネルギーに陽子からもらった分が加わると考えて
\begin{empheq}{align}
  \gamma_\mathrm{min} 
  &=
  \frac{1}{m_e c^2} ( kT_e + \zeta kT_p )\\
  &\approx
  \zeta \frac{kT_p}{m_e c^2} \:(\text{ignoring}\:kT_e)\\
  &=
  \zeta \frac{m_p}{2m_e}\frac{v_\mathrm{sh}^2}{c^2} \\
  &=
  33~\fzeta~\fvsh^{2}
  % 0.045 \left( 1 + \zeta\frac{m_p}{m_e} \right)
  % \left( \frac{v_\mathrm{sh}}{0.3c} \right)^2 \\
  % &\approx
  % 45\zeta 
  % \left( \frac{v_\mathrm{sh}}{0.3c} \right)^2
  % \;(\text{ignoring 1st term because}; 1\ll m_p/m_e \approx 10^3) \\
  % &=
  % 18\left( \frac{\zeta}{0.4} \right)\left( \frac{v_\mathrm{sh}}{0.3c} \right)^2
\end{empheq}
となる. 
% 最後の行で, \( 1 \ll m_p/m_e,\,\zeta \leq 1 \)より第1項を無視した.
% $\zeta$の値はKashiyama et al.(2018)より孫引き. (親はPark et al.(2015)?)

\section{シンクロトロン放射}
$\gamma_e$のLorenz factorをもつ電子からのシンクロトロン放射を考える.
放射の典型的な振動数$\nu_e$は
\begin{equation}
  \nu_e
  = \gamma_e^3\nu_B\sin\alpha
  = \frac{\gamma_e^2eB}{2\pi m_e c} \sin\alpha
\end{equation}
である(Rybicki \& Lightman (1979)).
ここで$\nu_B$はサイクロトロン振動数, $\alpha$はピッチ角である.
磁場がランダムな方向を向いていると仮定し, $\sin\alpha$の平均をとると
\begin{equation}
  \int_0^{\pi/2} \sin\alpha \dd{\alpha} = 1
\end{equation}
ゆえ
\begin{equation}
  \nu_e = \frac{\gamma_e^2eB}{2\pi m_e c}
\end{equation}
\( \gamma_e = \gamma_\mathrm{min} \)として値を代入すると
\begin{equation}
  \nu_e
  \approx
  \SI{11.2}{GHz}~
  \sqrt{\eta}~
  \fzeta^2~\fepsB^{1/2}~\feps^{1/2}~\fMx^{1/2}~
  \ftrec^{1/2}~\fvsh^{3/2}~\ftrise^{-1}~\ftdecay^{-1/2}
  % {\zeta_{\num{0.4}}}^2~
  % {\epsilon_{\mathrm{B},0.01}}^{1/2}~
  % {\epsilon_{0.1}}^{1/2}~
  % {M_{*,1.4}}^{1/2}~
  % {t_{\mathrm{rec},200}}^{1/2}~
  % {v_{\mathrm{sh},0.3}}^{3/2}~
  % {t_{\mathrm{rise},3}}^{-1}~
  % {\tau_{\num{10}}}^{-1/2}~
\end{equation}
% ひとまずKashiyama et al.(2018)にならって\( \zeta = 0.4,\,\epsilon_\mathrm{B} = 0.01 \)
% を入れてみると
% \begin{equation}
%   \nu_e \approx \SI{7.92}{GHz}
% \end{equation}

\section{Synchrotron cooling}
シンクロトロン放射した電子はその分エネルギーを失う.
あるLorentz factor$\gamma$をもつ電子がエネルギー密度$U_B$の磁場の中で単位時間に放射するエネルギー
(=電子のエネルギー損失率)は
\begin{equation}
  P(\gamma,B) = \frac{4}{3}\sigma_\mathrm{T}c\beta^2\gamma^2U_B
\end{equation}
と表せる(Rybicki\&Lightman, 1979).
いまの場合,
\begin{equation}
  \gamma = \gamma_\mathrm{min} = 33~\fzeta~\fvsh^{2}\:,
\end{equation}
\begin{equation}
  \beta^2 = 1 - \frac{1}{\gamma_\mathrm{min}^2} ~(=0.999)~ \approx 1
\end{equation}
\begin{equation}
  U_B
  =
  \frac{B^2}{8\pi}
  % =
  % \epsilon_\mathrm{B}
  % \frac{M_\mathrm{ej}v^2_\mathrm{sh}/2}{4\pi r^2_\mathrm{sh}v_\mathrm{sh}\tau}
  = \SI{0.53}{erg.em^{-3}}~
  \eta~\fepsB~\feps~\fMx~\fvsh^{-1}~\ftrec~\ftrise^{-2}~\ftdecay^{-1}
\end{equation}
なので
\begin{empheq}{align}
  P &= \SI{1.55e-11}{erg/s}
  ~\eta~\fzeta^2~\fepsB~\feps~\fMx~\fvsh^{3}~\ftrec~\ftrise^{-2}~\ftdecay^{-1}\\
  &= \SI{9.68e-6}{MeV/s}
  ~\eta~\fzeta^2~\fepsB~\feps~\fMx~\fvsh^{3}~\ftrec~\ftrise^{-2}~\ftdecay^{-1}
\end{empheq}
となる.
% ($\beta$に含まれる$v_\mathrm{sh}$の依存性は無視した).
%---betaのnormalized factorを残したversion
% \begin{empheq}{align}
%   P &= \SI{1.55e-11}{erg/s}
%   ~\eta~\fzeta^2~\fepsB~\feps~\fMx~\fvsh^{5}~\ftrec~\ftrise^{-2}~\ftdecay^{-1}\\
%   &= \SI{9.68e-6}{MeV/s}
%   ~\eta~\fzeta^2~\fepsB~\feps~\fMx~\fvsh^{5}~\ftrec~\ftrise^{-2}~\ftdecay^{-1}
% \end{empheq}
よって, シンクロトロン放射の寄与だけを考えた冷却時間は
\begin{equation}
  t_\mathrm{cool}
  \approx
  \frac{\gamma_\mathrm{min}m_ec^2}{P}
  =
  \SI{2.9e4}{min}
  ~\eta^{-1}~\fzeta^{-1}~\fepsB^{-1}~\feps^{-1}~\fMx^{-1}
  ~\fvsh^{-2}~\ftrec^{-1}~\ftrise^{2}~\ftdecay^{1}
\end{equation}
\end{document}
