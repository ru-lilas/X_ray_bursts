\documentclass{ltjsarticle}
%==============================================================================%
\usepackage{amsmath,amssymb}
\usepackage{bxpapersize} %用紙サイズを一致させるパッケージ
\usepackage{enumitem}
\usepackage{graphicx}
\usepackage[unicode=true]{hyperref}
\usepackage{mathtools}
\mathtoolsset{showonlyrefs}
\usepackage{siunitx, letltxmacro}
\AtBeginDocument{\RenewCommandCopy\qty\SI}
\LetLtxMacro{\svqty}{\qty}
\usepackage{physics}
\LetLtxMacro{\qty}{\svqty}
\usepackage{booktabs}
\usepackage{empheq}
\usepackage{titleps}
\newpagestyle{main}{
  \sethead[][][]
    {}{}{}
  \setfoot[][\thepage][]
    {}{\thepage}{}
}
\pagestyle{main}
\AtBeginDocument{\RenewCommandCopy\qty\SI}
%==============================================================================%
\begin{document}
  % \title{Assignment on 2024-10-29}
  % \author{KUMADA Ryota (C4SM3002)}
  % \date{}
  % \maketitle
%------------------------------------------------------------------------------%
% \section{電波放射するまでの流れ}
% \begin{enumerate}
%   \item X--ray burstでshockが形成される
%   \item shockが外側の物質を加熱する
% \end{enumerate}

\section{Accretion rateとejecta mass}
shock ejectaの質量$M_\mathrm{ej}$は, recurrence time \( t_\mathrm{rec} \approx \SI{200}{min} \)の間に降着する質量の$\eta$倍に等しいとする. 
accretion rateはEddington accretion rate $\dot{M}_{\mathrm{Edd}}$と仮定する.
ここで, $\dot{M}_{\mathrm{Edd}}$はEddington luminosity:
\begin{equation}
  L_\mathrm{Edd} \approx \SI{e38}{erg/s}\left( \frac{M}{M_\odot} \right)
\end{equation}
と, 輻射への変換効率$\epsilon$を用いて
\begin{equation}
  \dot{M}_{\mathrm{Edd}} = \frac{L_\mathrm{Edd}}{\epsilon c^2}
  =
  \SI{1.1e17}{g/s}~
  \left( \frac{\epsilon}{0.1} \right)^{-1}
  \left( \frac{M}{M_\odot} \right)
\end{equation}
と表せる. よって
\begin{equation}
  M_\mathrm{ej}
  =
  \eta\dot{M}_\mathrm{Edd} t_\mathrm{rec}
  =
  \SI{1.3e21}{g}~\eta
  \left( \frac{t_\mathrm{rec}}{\SI{200}{min}} \right)
  \left( \frac{\epsilon}{0.1} \right)
  \left( \frac{M}{M_\odot} \right)
\end{equation}

\section{電波立ち上がり時のshock radius}
X--ray burstの瞬間に\(t=0\)にshockが立ち, \(v_\mathrm{sh} = 0.3c\)の速さで走ると仮定する.
観測によれば, X--ray burstが起きてから電波が立ち上がるまでの時間は\( t_\mathrm{rise} \approx \SI{3}{min} \)なので
\begin{equation}
  r_\mathrm{sh} = v_\mathrm{sh}t_\mathrm{rise} 
  =
  \SI{1.6e12}{cm}~\left(\frac{v_\mathrm{sh}}{0.3c}\right)\left( \frac{t_\mathrm{rise}}{\SI{3}{min}} \right)
  =
  \SI{0.11}{AU}~\left(\frac{v_\mathrm{sh}}{0.3c}\right)\left( \frac{t_\mathrm{rise}}{\SI{3}{min}} \right)
\end{equation}

\section{磁場の強さ}
collisionless--shockはそのエネルギーの一部を磁場へ渡す. 変換効率を$\epsilon_B$とすると
\begin{equation}
  \epsilon_\mathrm{B}\frac{M_\mathrm{ej}v^2_\mathrm{sh}/2}{4\pi r^2_\mathrm{sh}v_\mathrm{sh}\tau}
  =
  \frac{B^2}{8\pi}
\end{equation}
ここで\( \tau \approx \SI{10}{min} \)はX--ray burstが起きてから電波が減衰するまでの時間である.
$B$について解くと
\begin{empheq}{align}
  B^2
  &=
  \epsilon_B\frac{M_\mathrm{ej}v_\mathrm{sh}}{r^2_\mathrm{sh}\tau}
  =
  \eta\epsilon_B
  \frac{\dot{M}_\mathrm{Edd}t_\mathrm{rec}}{v_\mathrm{sh} t^2_\mathrm{rise}\tau}\\
  &\approx
  \SI{7.6e3}{erg.cm^{-3}}~
  \eta\epsilon_\mathrm{B}\left( \frac{\epsilon}{0.1} \right) \left( \frac{M}{M_\odot} \right)
  \left( \frac{t_\mathrm{rec}}{\SI{200}{min}} \right)
  \left( \frac{v_\mathrm{sh}}{0.3c} \right)^{-1}
  \left( \frac{t_\mathrm{rise}}{\SI{3}{min}} \right)^{-2}
  \left( \frac{\tau}{\SI{10}{min}} \right)
\end{empheq}
\begin{equation}
  B
  \approx
  \SI{87}{G}~
  \sqrt{\eta\epsilon_\mathrm{B}}
  \left( \frac{\epsilon}{0.1} \right)^{1/2}
  \left( \frac{M}{M_\odot} \right)^{1/2}
  \left( \frac{t_\mathrm{rec}}{\SI{200}{min}} \right)^{1/2}
  \left( \frac{v_\mathrm{sh}}{0.3c} \right)^{-1/2}
  \left( \frac{t_\mathrm{rise}}{\SI{3}{min}} \right)^{-1}
  \left( \frac{\tau}{\SI{10}{min}} \right)^{1/2}
\end{equation}

\section{周辺ガスの密度}
\( \tau \approx \SI{10}{min} \)の間にshockが掃いた領域にあるガスの総質量は$M_\mathrm{ej}$に等しいと仮定する. すなわち
\begin{equation}
  4\pi r^2_\mathrm{sh}\rho_\mathrm{amb}v_\mathrm{sh}\tau = M_\mathrm{ej}
\end{equation}
$\rho_\mathrm{amb}$は周辺ガスの質量密度である. $\rho_\mathrm{amb}$について解くと
\begin{empheq}{align}
  \rho_\mathrm{amb}
  &=
  \frac{M_\mathrm{ej}}{4\pi r^2_\mathrm{sh}v_\mathrm{sh}\tau}\\
  &\approx
  \SI{7.5e-18}{g.cm^{-3}}
  \left( \frac{\epsilon}{0.1} \right)
  \left( \frac{M}{M_\odot} \right)
  \left( \frac{t_\mathrm{rec}}{\SI{200}{min}} \right)
  \left( \frac{v_\mathrm{sh}}{0.3c} \right)^{-3}
  \left( \frac{t_\mathrm{rise}}{\SI{3}{min}} \right)^{-2}
  \left( \frac{\tau}{\SI{10}{min}} \right)^{-1}
\end{empheq}

\section{Power--lawの比例係数}
電子分布はpower lawに従うとする.
\begin{equation}
  \dv{N}{\gamma} = C\gamma^{-p}\quad(\text{for}\;\gamma_\mathrm{min} < \gamma < \gamma_\mathrm{max})
\end{equation}
shockはそのエネルギーの一部を周囲の電子へ受け渡す. 効率を$\epsilon_\mathrm{e}$とおくと
\begin{equation}
  \epsilon_\mathrm{e} \frac{M_\mathrm{ej}}{2}v^2_\mathrm{sh}
  =
  \int_{\gamma_\mathrm{min}}^{\gamma_\mathrm{max}} \gamma m_\mathrm{e}c^2 \dv{N}{\gamma}\dd{\gamma}
  =
  m_\mathrm{e}c^2 \frac{C}{2-p}
  \left[ \gamma^{2-p} \right]_{\gamma_\mathrm{min}}^{\gamma_\mathrm{max}}
\end{equation}
となる. $C$について解くと
\begin{empheq}{align}
  C
  &=
  \epsilon_\mathrm{e} \frac{M_\mathrm{ej}v^2_\mathrm{sh}}{2m_\mathrm{e}c^2}
  (2-p) \left( \gamma_\mathrm{max}^{2-p} - \gamma_\mathrm{min}^{2-p} \right)^{-1}\\
  &=
  \num{6.6e46}\eta\epsilon_\mathrm{e}~
  (2-p) \left( \gamma_\mathrm{max}^{2-p} - \gamma_\mathrm{min}^{2-p} \right)^{-1}
  \left( \frac{\dot{M}}{\dot{M}_\mathrm{Edd}} \right)
  \left( \frac{t_\mathrm{rec}}{\SI{200}{min}} \right)
  \left( \frac{v_\mathrm{sh}}{0.3c} \right)^2
\end{empheq}

shockで加速される電子のエネルギー範囲($\gamma_\mathrm{min},\,\gamma_\mathrm{max}$)が
分かれば, $C$の値を見積もることができる.

\section{放射電子の最小エネルギー}
プラズマガスはshockからエネルギーを受け取り熱化される. 電子の熱エネルギー$kT_e$が, $v_\mathrm{sh}$で走る電子の運動エネルギーに等しいと仮定すると
\begin{equation}
  kT_e
  \approx \frac{m_e}{2}v_\mathrm{sh}^2
  \approx 0.045 m_e c^2 \left( \frac{v_\mathrm{sh}}{0.3c} \right)^2
\end{equation}

プラズマが電子と陽子のみから構成されていると仮定する. 陽子の熱化についても電子と同様に考えると,
各々のLorentz factorはいずれも
\begin{equation}
  \gamma_\mathrm{th} = \frac{kT_i}{m_i c^2} \approx 0.045\quad (i = \mathrm{e,\,p})
\end{equation}
となる.

熱化された陽子と電子とが互いにエネルギーをやりとりできるのであれば,
電子が陽子の熱エネルギーの一部を受け取れるはずである.
電子が受け取る, 陽子の熱エネルギーの割合を$\zeta$とすると
\begin{empheq}{align}
  \gamma_e 
  &=
  \frac{1}{m_e c^2} ( kT_e + \zeta kT_p ) \\
  &=
  0.045 \left( 1 + \zeta\frac{m_p}{m_e} \right)
  \left( \frac{v_\mathrm{sh}}{0.3c} \right)^2 \\
  &\approx
  45\zeta 
  \left( \frac{v_\mathrm{sh}}{0.3c} \right)^2
  \;(\text{using}\; m_p/m_e \approx 10^3) \\
  % &=
  % 18\left( \frac{\zeta}{0.4} \right)\left( \frac{v_\mathrm{sh}}{0.3c} \right)^2
\end{empheq}
となる. 
% $\zeta$の値はKashiyama et al.(2018)より孫引き. (親はPark et al.(2015)?)

\section{シンクロトロン放射}
$\gamma_e$のLorenz factorをもつ電子からのシンクロトロン放射を考える.
放射の典型的な振動数$\nu_e$は
\begin{equation}
  \nu_e
  = \gamma_e^3\nu_B\sin\alpha
  = \frac{\gamma_e^2eB}{2\pi m_e c} \sin\alpha
\end{equation}
である(Rybicki \& Lightman (1979)).
ここで$\nu_B$はサイクロトロン振動数, $\alpha$はピッチ角である.
磁場がランダムな方向を向いていると仮定し, $\sin\alpha$の平均をとると
\begin{equation}
  \int_0^{\pi/2} \sin\alpha \dd{\alpha} = 1
\end{equation}
ゆえ
\begin{equation}
  \nu_e = \frac{\gamma_e^2eB}{2\pi m_e c}
\end{equation}
値を代入すると
\begin{equation}
  \nu_e
  \approx
  \SI{495}{GHz}~
  \zeta\sqrt{\eta\epsilon_\mathrm{B}}
  % \left( \frac{\zeta}{0.4} \right)
  \left( \frac{\epsilon}{0.1} \right)^{1/2}
  \left( \frac{M}{M_\odot} \right)^{1/2}
  \left( \frac{t_\mathrm{rec}}{\SI{200}{min}} \right)^{1/2}
  \left( \frac{v_\mathrm{sh}}{0.3c} \right)^{3/2}
  \left( \frac{t_\mathrm{rise}}{\SI{3}{min}} \right)^{-1}
  \left( \frac{\tau}{\SI{10}{min}} \right)^{1/2}
\end{equation}
% ひとまずKashiyama et al.(2018)にならって\( \zeta = 0.4,\,\epsilon_\mathrm{B} = 0.01 \)
% を入れてみると
% \begin{equation}
%   \nu_e \approx \SI{7.92}{GHz}
% \end{equation}
\end{document}
