\documentclass{ltjsarticle}
%==============================================================================%
\usepackage{amsmath,amssymb}
% \let\qty\relax
\usepackage{bxpapersize} %用紙サイズを一致させるパッケージ
\usepackage{enumitem}
\usepackage{graphicx}
\usepackage[unicode=true]{hyperref}
\usepackage{mathtools}
\mathtoolsset{showonlyrefs}
% \usepackage{physics}
\usepackage{siunitx}
\usepackage{booktabs}
\usepackage{empheq}
\usepackage{titleps}
\newpagestyle{main}{
  \sethead[][][]
    {}{}{}
  \setfoot[][\thepage][]
    {}{\thepage}{}
}
\pagestyle{main}
%==============================================================================%
\begin{document}
  % \title{Assignment on 2024-10-29}
  % \author{KUMADA Ryota (C4SM3002)}
  % \date{}
  % \maketitle
%------------------------------------------------------------------------------%
\section{電波放射するまでの流れ}
\begin{enumerate}
  \item X--ray burstでshockが形成される
  \item shockが外側の物質を加熱する
\end{enumerate}

\section{見積もり}
ejectaの質量$M_\mathrm{ej}$は, recurrence timeの間にaccretionで落ちてきた質量の$\eta$倍に等しいと見積もってみる。
accretion rateはEddington accretion rate$\dot{M}_{\mathrm{Edd}}$に等しいと仮定する。ここで, $\dot{M}_{\mathrm{Edd}}$はEddington luminosity:
\begin{equation}
  L_\mathrm{Edd} \approx \SI{e38}{erg/s}\left( \frac{M_{\mathrm{NS}}}{M_\odot} \right),
\end{equation}

\end{document}
